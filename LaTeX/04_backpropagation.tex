\newpage
\thispagestyle{empty}
\section{Backpropagation}\label{sec:backpropagation}   

\vspace{1cm}
\begin{tcolorbox}[title={Inhalt}]
Die Backpropagation umfasst folgende Elemente:
\begin{itemize}
\item Wie lernen Neuronale Netze?
\item Grundidee Backpropagation
\item Fehlerfunktion finden
\item Gewichtsanpassung
\begin{quotation}
Eine Einleitung muss auch durch die Arbeit führen. Sie muss dem Leser helfen, sich in der Arbeit und ihrer Struktur zu Recht zu finden. Für jedes Kapitel sollte eine ganz kurze Inhaltsangabe gemacht werden und ggf. motiviert werden, warum es geschrieben wurde. Oft denkt sich ein Autor etwas bei der Struktur seiner Arbeit, auch solche Beweggründe sind dem Leser zu erklären\footnote{\cite{BBoJ}, S. 6}:. 
\end{quotation}
\end{itemize}
\end{tcolorbox}

\subsection{Wie lernen Neuronale Netze?}\label{subsec:backpropagation:lernen_nn}

\subsection{Grundidee Backpropagation}\label{subsec:backpropagation:grundiee}
\subsubsection{Fehlerrückführung}\label{subsec:backpropagation:fehlerrueckfuehrung}
  %\input{}
  TEXT FOLGT...

\subsubsection{Methode mit Matrizenmultiplikation}\label{subsec:backpropagation:matrizen}
  %\input{}
  TEXT FOLGT...

\subsubsection{Forward / Backward Phase erklären}\label{subsec:backpropagation:forward_backward}
  %\input{}
  TEXT FOLGT...

\subsection{Fehlerfunktion finden}\label{subsec:backpropagation:fehlerfunktion}

\subsection{Gewichtsanpassung (Maybe)}\label{subsec:backpropagation:gewichtsanpassung}
