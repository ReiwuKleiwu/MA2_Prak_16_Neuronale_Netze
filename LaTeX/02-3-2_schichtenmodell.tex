\subsubsection{Schichtenmodell}\label{subsec:neuronen:schichtenmodell}
Neuronale Netze sind in einer Struktur aus mehreren aufeinanderfolgenden Schichten aufgebaut.
Es existieren jedoch auch Netze, die nur aus einer einzigen Schicht bestehen, die sogenannten Perzeptrons, wobei alle Neuronen dieser Schicht mit allen Eingaben verbunden sind.
Diese sind aber verhältnismäßig wenig leistungsfähig und können keine komplexen Zusammenhänge verarbeiten.
Um komplexere Merkmale in den Eingabedaten zu erkennen, verwendet man Netze aus mehreren Schichten, was auch als ein "feed-forward" neuronales Netz bezeichnet wird.
Dabei unterscheidet man grundlegend zwischen drei verschiedenen Arten von Schichten, der Eingabeschicht, den versteckten Schichten ("hidden layer") und der Ausgabeschicht.
Die einzige Aufgabe der Eingabeschicht ist es, die Daten der Eingabevariablen darzustellen und an die folgenden Neuronen weiterzugeben, was auch bedeutet, dass diese Neuronen keine Aktivierungsfunktionen verwenden.
Die Werte der Neuronen, die die Ausgabeschicht bilden, sind auch die Werte, die als Ausgabe des Netzes gelten.
