% Erklärung über die selbständige Abfassung der Arbeit  
\newpage
\pagestyle{empty}
\section*{Erklärung über die selbständige\\Abfassung der Arbeit} % \section*{...}: das *-Symbol erlaubt, dass dieser
% Gliederungspunkt nicht ins Inhaltsverzeichnis aufgenommen wird
\addcontentsline{toc}{section}{Erklärung über die selbständige Abfassung der Arbeit}
Ich versichere, die von mir vorgelegte Arbeit selbständig verfasst zu haben.
Alle Stellen, die wörtlich oder sinngemäß aus veröffentlichten oder nicht veröffentlichten Arbeiten anderer entnommen sind,
habe ich als entnommen kenntlich gemacht.\\ 
Sämtliche Quellen und Hilfsmittel, die ich für die Arbeit benutzt habe, sind
angegeben. Die Arbeit hat mit gleichem Inhalt bzw. in wesentlichen Teilen noch keiner anderen Prüfungsbehörde vorgelegen.\\\\
\begin{tabular}{cp{7cm}}
    & \\ 
    & \\ \hline
    \small (Ort, Datum, Unterschrift) & \normalsize \\
    \end{tabular}
    
    %<MERKKASTEN> (für die eigene Verwendung bitte entfernen
    \vspace{1cm}
\begin{tcolorbox}[title={Hinweise zur obigen \textit{Erklärung}}]
\begin{itemize}
\item Bitte verwenden Sie nur die Erklärung, die Ihnen Ihr \textbf{Prüfungsservice} vorgibt. Ansonsten könnte es passieren, dass Ihre Abschlussarbeit nicht angenommen wird. Fragen Sie im Zweifelsfalle bei Ihrem Prüfungsservice nach.
\item Sie müssen \textbf{alle abzugebende Exemplare} Ihrer Abschlussarbeit unterzeichnen. Sonst wird die Abschlussarbeit nicht akzeptiert. 
\item Ein \textbf{Verstoß} gegen die unterzeichnete \textit{Erklärung} kann u.\,a. die Aberkennung Ihres akademischen Titels zur Folge haben.
\end{itemize}  
\end{tcolorbox}
%</MERKKASTEN>   