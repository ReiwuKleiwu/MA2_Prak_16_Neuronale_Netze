Verlustfunktionen (Loss Functions) oder auch bekannt als Kostenfunktionen bewerten den berechneten Wert $\hat{y}$ eines neuralen 
Netzes in Relation zum erwarteten Wert $y$. Um die Genauigkeit der vorhergesagten Werte zu optimieren,
werden Verlustfunktionen beim Trainieren verwendet \cite{GBCL18}. Auch wie bei den Aktivierungsfunktionen gibt es
viele Verlustfunktionen, welche erheblichen Einfluss auf das Ergebnis eines neuralen Netzes haben. Die Wahl der Verlustfunktionen
hängt von der Architektur des neuralen Netzes und der Ausgabewerte der Output-Nodes ab. Für ein einfachen Perceptron wird eine 
Verlustfunktion der Form $(y - \hat{y})^2$ verwendet. 
Alternative Verlustfunkionen währen z.B. die hingle-loss-Funktion $L = \max\{0,1-y\cdot\hat{y}\}$, welche genutzt werden kann um z.B. die 
Support Vector Machine Lernmethode zu implementieren. Für diese Verlustfunkion muss $y \in \{-1,+1\}$ gelten und $\hat{y}$ muss reele Zahlen 
darstellen \cite{CA18}.\\
Für probabilistische Vorhersagen bei denen $y \in \{-1,+1\}$ gilt und $\hat{y}$ eine reele Zahl ist,  eignet sich 
$L = \log(1 + \exp(-y \cdot \hat{y}))$ als Verlustfunktion. Diese Art von Verlustfuntktion implementiert die logisitical Regression 
Machine Learning Methode \cite{CA18}.
