\section{Implementierung mit Keras}\label{sec:code} 
Die untenstehenden Code Beispiele orientieren sich am Leitfaden von Maximilian Wittman \cite{WM20}
\subsection{Perceptron}
Das folgende Perceptron ist ein neurales Netz, welches nur unterscheiden kann ob ein Bild eine 5 ist oder nicht. Die binäre Entscheidung wird mit Hilfe
der Aktivierungsfunktion sigmoid getroffen.
\begin{lstlisting}[basicstyle=\ttfamily\footnotesize]
    from keras.datasets import mnist
    from keras.models import Sequential
    from keras.layers import Dense, Flatten, Conv2D, MaxPooling2D
    from keras.utils import to_categorical
    from PIL import Image
    import numpy as np
    
    # Laden Sie den MNIST-Datensatz herunter
    (train_images, train_labels), (test_images, test_labels) = mnist.load_data()
    
    # Wir muessen die Bilder neu formatieren, damit sie mit Keras funktionieren
    train_images = train_images.reshape((60000, 28, 28, 1))
    test_images = test_images.reshape((10000, 28, 28, 1))
    
    # Normalisieren Sie die Bilder auf Werte zwischen 0 und 1
    train_images, test_images = train_images / 255.0, test_images / 255.0
    
    # aendere Label zu 1, wenn 5 sonst 0
    train_labels = np.where(train_labels == 5, 1, 0)
    test_labels = np.where(test_labels == 5, 1, 0)
    
    # Erstellen Sie das Modell
    model = Sequential()
    # Bild in Vektor umwandeln und Input Layer mit 28*28*1 Neuronen erstellen
    model.add(Flatten(input_shape=(28, 28, 1)))
    # Output Layer mit einem Neuron erstellen
    model.add(Dense(1, activation='sigmoid'))
    # Kompilieren Sie das Modell
    model.compile(optimizer='adam',
                  loss='binary_crossentropy',
                  metrics=['accuracy'])
    # Trainieren Sie das Modell
    model.fit(train_images, train_labels, epochs=50, batch_size=64)
    # Evaluieren Sie das Modell
    test_loss, test_acc = model.evaluate(test_images, test_labels)
    print('Test accuracy:', test_acc)
    # Test accuracy konvergiert gegen 0.977
\end{lstlisting}


\subsection{Neurales Netz mit hidden Layer}
Das Ziel des neuralen Netzes aus dieser Implementation ist Ziffern von 0-9 aus Bildern zu identifizieren. Sie hat eine Input Layer, hidden Layer 
und eine Output Layer mit 10 Neuronen.
\begin{lstlisting}[basicstyle=\ttfamily\footnotesize]
    from keras.datasets import mnist
    from keras.models import Sequential
    from keras.layers import Dense, Flatten, Conv2D, MaxPooling2D
    from keras.utils import to_categorical
    from PIL import Image
    import numpy as np
    
    # Laden Sie den MNIST-Datensatz herunter
    (train_images, train_labels), (test_images, test_labels) = mnist.load_data()
    
    # Wir muessen die Bilder neu formatieren, damit sie mit Keras funktionieren
    train_images = train_images.reshape((60000, 28, 28, 1))
    test_images = test_images.reshape((10000, 28, 28, 1))
    
    # Normalisieren Sie die Bilder auf Werte zwischen 0 und 1
    train_images, test_images = train_images / 255.0, test_images / 255.0
    
    # Konvertieren Sie die Labels in kategorische Daten
    train_labels = to_categorical(train_labels)
    test_labels = to_categorical(test_labels)
    
    # Erstellen Sie das Modell
    model = Sequential()
    
    # Fuegen Sie das erste Convolutional Layer hinzu
    model.add(Flatten(input_shape=(28, 28, 1)))
    
    # Fuegen Sie ein MaxPooling Layer hinzu
    model.add(Dense(64, activation='relu'))
    
    # Fuegen Sie das Output Layer hinzu. Da wir MNIST verwenden, haben wir 10 Nodes.
    model.add(Dense(10, activation='softmax'))
    
    # Kompilieren Sie das Modell
    model.compile(optimizer='adam',
                  loss='categorical_crossentropy',
                  metrics=['accuracy'])
    
    # Trainieren Sie das Modell
    model.fit(train_images, train_labels, epochs=50, batch_size=64)
    
    # Evaluieren Sie das Modell
    test_loss, test_acc = model.evaluate(test_images, test_labels)
    print('Test accuracy:', test_acc)
    # Test accuracy konvergiert gegen 1.000
\end{lstlisting}