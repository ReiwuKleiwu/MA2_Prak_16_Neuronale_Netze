In einem einfachen neuronalen Netz wie einem Multilayer-Perceptron sind alle Neuronen der einen Schicht jeweils mit allen Neuronen der folgenden Schicht verbunden.
Jeder Verbindung zwischen zwei Neuronen ist jeweils ein sogenanntes Weight zugeordnet.


\subsubsection{Weights}\label{subsec:neuronen:verknuepfung_neuronen:Weights}
  Die Verknüpfungen zwischen Neuronen leiten Signale nicht einfach unverändert an das nächste Neuron weiter.
  Die Weights, die jeder Verbindung zwischen Neuronen zugewiesen sind können die Signale sowohl verstärken als auch abschwächen, wobei ein höheres Weight zu einer Verstärkung führt.
  Weights sind der Faktor innerhalb des neuronalen Netzes, der während des Trainingsvorgangs verändert wird.
  Jedem Neuron ist dabei ein eigenes Weight zugeordnet. Die Gewichtungen dienen also dazu, dass das neuronale Netz überhaupt lernen kann.
  Unwichtige Verbindungen werden während des Lernprozesses abgeschwächt und einige können dadurch sogar fast gänzlich blockiert werden, wohingegen andere Verbindungen verstärkt werden.
  Dadurch ist es zum Beispiel möglich, dass ein Neuron nur die Werte von einzelnen anderen Neuronen verarbeitet und andere ignoriert.
  Dies hilft den Neuronen dabei, Muster in den Eingabedaten zu erkennen. 