
\section{Einleitung Neuronale Netze}\label{Einleitung}   
NN (wofür, Einsatzgebiete, Grobes Prinzip). EINFÜGEN

\vspace{1cm}
\begin{tcolorbox}[title={Inhalt}]
Die Einleitung umfasst folgende Elemente\footnote{Vgl. u.a. \cite{BBoJ}, S. 5-6}:
\begin{itemize}
\item Wofür benötigt man Neuronale Netze
\item Einsatzgebiete
\item Grobes Prinzip
\begin{quotation}
Eine Einleitung muss auch durch die Arbeit führen. Sie muss dem Leser helfen, sich in der Arbeit und ihrer Struktur zu Recht zu finden. Für jedes Kapitel sollte eine ganz kurze Inhaltsangabe gemacht werden und ggf. motiviert werden, warum es geschrieben wurde. Oft denkt sich ein Autor etwas bei der Struktur seiner Arbeit, auch solche Beweggründe sind dem Leser zu erklären\footnote{\cite{BBoJ}, S. 6}:. 
\end{quotation}
\end{itemize}
\end{tcolorbox}

\subsection{Wofür benötigt man Neuronale Netze}\label{Einleitung} 

\subsection{Einsatzgebiete}\label{Einleitung}

\subsection{Grobes Prinzip}\label{Einleitung}


%Neuronen 
\newpage  
\section{Was sind Neuronen?}\label{kap_Neuronen}  
 Bitte Einfügen (Arten von Neuronen, Funktionsweise)
 \begin{tcolorbox}[title={Inhalt}]
 \begin{itemize} 
   \item Arten von Neuronen
   \item Funktionsweise
   \item Aktivierungsfunktion
   \item Schichtenmodell
 \end{itemize} 
 \end{tcolorbox}

 \subsection{Arten von Neuronen}\label{arten_von_neuronen}
  %\input{}
  TEXT FOLGT... 


\newpage
\subsection{Funktionsweise}\label{Funktionsweise}
  %\input{}
  TEXT FOLGT... 
 

\newpage
\subsubsection{Aktivierungsfunktion}\label{Aktivierungsfunktion}
  TODO
\paragraph{Linear activation}
TODO\\
$\Phi(v) = v$
\paragraph{Nicht lineare Aktivierung}
TODO
\subparagraph{Sign activation}
TODO\\
$\Phi(v) = sign(v)$
\subparagraph{Sigmoid activation}
TODO\\
$\Phi(v) = \frac{1}{1 + e^{-v}}$
\subparagraph{Tanh activation}
TODO\\
$\Phi(v) = \frac{e^{2v} - 1}{e^{2v} + 1}$\\
$tanh(v) = 2 \cdot sigmoid(2v) - 1$
\paragraph{Piecewise lineare Aktivierung}
\subparagraph{ReLU}
TODO\\
$\Phi(v) = max\{v,0\}$
\subparagraph{hard tanh activations}
TODO\\
$\Phi(v) = max\{min[v,1],-1\}$
  TEXT FOLGT... 


\newpage
\subsubsection{Schichtenmodell}\label{Schichtenmodell}
  %\input{}
  TEXT FOLGT... 

\subsubsection{Loss-Function}

\newpage 
\subsection{Wie sind Neuronen miteinander verknüpft}\label{kap_verknuepfung_neuronen}  
%\input{}
Hier sollen die Weights erklärt werden

\subsubsection{Weights}\label{Weights}
  %\input{}
  TEXT FOLGT... 

\subsection{Fehler / Backpropagation Einführung}\label{kap_fehler_backpropagation}
Nur eine sehr knappe Einführung, da eigenes Kapitel für dieses Thema reserviert

