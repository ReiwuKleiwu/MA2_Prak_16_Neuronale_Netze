\newpage
\thispagestyle{empty}
\section{Gradientenverfahren}\label{sec:gradientenverfahren}   
\begin{tcolorbox}[title={Inhalte des \textit{Gradientenverfahren}}]
  \begin{itemize}
    \item Wofür braucht mann das Gradientenverfahren?
    \item Grundkonzepte des Gradientenabstiegsverfahren
    \item Gefährliche Fehlerquellen
  \begin{quotation}
      %Zum Verstädnis, evtl löschen oder ab ändern.
      Dieser Abschnitt führt den Leser durch die aufgezählten Themen.
      Eine Einleitung muss auch durch die Arbeit führen. Sie muss dem Leser helfen, sich in der Arbeit und ihrer Struktur zu Recht zu finden.
      Für jedes Kapitel sollte eine ganz kurze Inhaltsangabe gemacht werden und ggf. motiviert werden, warum es geschrieben wurde. 
      Oft denkt sich ein Autor etwas bei der Struktur seiner Arbeit, auch solche Beweggründe sind dem Leser zu erklären. 
  \end{quotation}
  \end{itemize}
\end{tcolorbox}


\subsection{Wofür braucht mann das Gradientenverfahren?}\label{subsec:gradientenverfahren:wofuer}
\subsubsection{Was ist ein Gradient?}\label{subsec:gradientenverfahren:was_ist_gradient}
  %\input{}
  TEXT FOLGT...


\subsection{Grundkonzepte des Gradientenverfahrens}\label{subsec:gradientenverfahren:grundkonzepte}
\subsubsection{Grundlage für den Fehlerrückführungs-Algorithmus - Wozu dient er?}\label{subsec:gradientenverfahren:grundlage_fehlerrueckfuehrungsalg}
  %\input{}
  TEXT FOLGT...

\subsubsection{Wie funktioniert das Gradientenverfahren?}\label{subsec:gradientenverfahren:wie_funktioniert}
  %\input{}
  TEXT FOLGT...

\subsubsection{Mehrere Dimensionen}\label{subsec:gradientenverfahren:mehrere_dimensionen}
  %\input{}
  TEXT FOLGT...



\subsection{Gefährliche Fehlerquellen}\label{subsec:gradientenverfahren:fehlerquellen}
\subsubsection{Befindet man sich wirklich im globalen Minimum?}\label{subsec:gradientenverfahren:fehlerquellen_globalen_minimum}
  %\input{}
  TEXT FOLGT...

\subsubsection{Steckt man in einem lokalen Minimum fest?}\label{subsec:gradientenverfahren:fehlerquellen_lokalen_minimum}
  %\input{}
  TEXT FOLGT...

\subsubsection{Wie löst man dieses Problem?}\label{subsec:gradientenverfahren:fehlerquellen_problem_loesen}
  %\input{}
  TEXT FOLGT...


