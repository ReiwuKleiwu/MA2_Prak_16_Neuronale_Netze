\newpage
\thispagestyle{empty}
\section{Gradientenverfahren}\label{sec:gradientenverfahren}   
\begin{tcolorbox}[title={Inhalte des \textit{Gradientenverfahren}}]
  \begin{itemize}
    \item Wofür braucht mann das Gradientenverfahren?
    \item Grundkonzepte des Gradientenabstiegsverfahren
    \item Gefährliche Fehlerquellen
  \begin{quotation}
      Das Kapitel Gradientenverfahren stellt die Grundlagen dar, die für das Verständnis des Lernprozesses eines neuronalen Netzwerks im nachfolgenden Kapitel erforderlich sind.
  \end{quotation}
  \end{itemize}
\end{tcolorbox}


\subsection{Wofür braucht mann das Gradientenverfahren?}\label{subsec:gradientenverfahren:wofuer}
\subsubsection{Was ist ein Gradient?}\label{subsec:gradientenverfahren:was_ist_gradient}
  %\input{}
  TEXT FOLGT...


\subsection{Grundkonzepte des Gradientenverfahrens}\label{subsec:gradientenverfahren:grundkonzepte}
\subsubsection{Grundlage für den Fehlerrückführungs-Algorithmus - Wozu dient er?}\label{subsec:gradientenverfahren:grundlage_fehlerrueckfuehrungsalg}
  %\input{}
  TEXT FOLGT...

\subsubsection{Wie funktioniert das Gradientenverfahren?}\label{subsec:gradientenverfahren:wie_funktioniert}
  %\input{}
  TEXT FOLGT...

\subsubsection{Mehrere Dimensionen}\label{subsec:gradientenverfahren:mehrere_dimensionen}
  %\input{}
  TEXT FOLGT...



\subsection{Gefährliche Fehlerquellen}\label{subsec:gradientenverfahren:fehlerquellen}
\subsubsection{Befindet man sich wirklich im globalen Minimum?}\label{subsec:gradientenverfahren:fehlerquellen_globalen_minimum}
  %\input{}
  TEXT FOLGT...

\subsubsection{Steckt man in einem lokalen Minimum fest?}\label{subsec:gradientenverfahren:fehlerquellen_lokalen_minimum}
  %\input{}
  TEXT FOLGT...

\subsubsection{Wie löst man dieses Problem?}\label{subsec:gradientenverfahren:fehlerquellen_problem_loesen}
  %\input{}
  TEXT FOLGT...


