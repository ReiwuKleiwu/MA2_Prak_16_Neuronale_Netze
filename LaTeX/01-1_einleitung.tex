\begin{tcolorbox}[title={Inhalt}]
  \begin{quotation}\noindent
      Dieses Kapitel liefert einen Einstieg in neuronale Netze. Es werden an Beispielen unterschiedliche Anwendungsfälle anschaulich gemacht.
      Ebenso soll bereits eine grobe Idee näher gebracht werden, wie ein neuronales Netz aufgebaut ist.
      Insgesamt werden in diesem Kapitel bereits zahlreiche grundlegende Begriffe erklärt, die in den nachfolgenden Kapiteln benötigt werden.
  \end{quotation}
  \begin{itemize}  
    \item Wofür benötigt man Neuronale Netze?
    \item Einsatzgebiete
    \item Grobes Prinzip
  \end{itemize}
\end{tcolorbox}

\subsection{Wofür benötigt man Neuronale Netze?}\label{subsec:einleitung_nn:wofuer_nn}
Mit neuronalen Netzen lassen sich Probleme lösen, die mit herkömmlichen Algorithmen nur schwer bis gar nicht lösbar sind. Neuronale Netze sind grundsätzlich in der Lage, nicht lineare Abbildungen anzunähern, Beispiele folgen weiter unten (siehe \ref*{subsec:einleitung_nn:einsatzgebiete}).
Der große Vorteil neuronaler Netze liegt darin, dass sie selbstständig lernen können, komplexe Muster in großen Datenmengen zu erkennen, wodurch man die entsprechenden Algorithmen nicht mehr von Hand schreiben muss. \cite{CA18}

\bigbreak\noindent
Aufgrund ihrer Fähigkeit, schnell und effizient komplexe Muster zu erkennen, können neuronale Netze auch Aufgaben lösen, für die man normalerweise Menschen benötigen würde, da herkömmliche Algorithmen diese Probleme nicht zuverlässig genug lösen können oder der Entwicklungsaufwand unverhältnismäßig groß in Relation zum Nutzen wäre. \cite{KSH17}

\bigbreak\noindent
Häufig sind Muster zu komplex, um sie mit einem klassischen Algorithmus erfassen zu können oder sie sind schlecht von Hand mathematisch formulierbar, wie zum Beispiel die Bilderkennung.
Mit Hilfe eines neuronalen Netzes kann man aber dennoch mit relativ geringem Aufwand ein Netz trainieren, das in der Lage ist, solche Bilderkennungsaufgaben zu erledigen \cite{KSH17}.

\bigbreak\noindent
Neuronale Netze sind häufig auch zuverlässiger als klassische Algorithmen, da diese manchmal bestimmte Sonderfälle nicht abdecken, wohingegen neuronale Netze resistenter gegen fehlerhafte Daten und Rauschen sind und besser mit Sonderfällen umgehen können.

\subsection{Einsatzgebiete}\label{subsec:einleitung_nn:einsatzgebiete}
Neuronale Netze sind äußerst vielfältig und lassen sich für verschiedenste Anwendungen anpassen und trainieren.
Zu den häufigsten Anwendungsfällen zählen die Bilderkennung und die Verarbeitung von auditiven Daten.

\bigbreak\noindent
Die Bilderkennung mittels neuronaler Netze lässt sich beispielsweise dazu einsetzen, medizinische Diagnosen durchzuführen, da diese Netze fähig sind, komplexe Krankheitsbilder zu erkennen \cite{VTY16}.
Auch kann die Bilderkennung genutzt werden, um die Position von Personen oder anderen Objekten auf einem Kamerabild zu identifizieren, was zur Realisierung von autonomem Fahren nützlich ist \cite{BMDT16}.
Die Bilderkennung mittels neuronaler Netze findet auch in der Industrie Anwendung, um hergestellte Produkte automatisiert auf Mängel zu prüfen und auszusortieren \cite{WCQS18}.

\bigbreak\noindent
Neuronale Netze können auch in Webseiten und Apps eingesetzt werden, um das Nutzerverhalten und Nutzerinhalte zu analysieren und den Nutzern so relevante Inhalte vorschlagen zu können. \cite{YHCE18}

\bigbreak\noindent
Grundsätzlich sind neuronale Netze in der Lage, beliebige Arten von Eingabedaten zu verarbeiten.
So ist es auch möglich, Audio-Daten zu verarbeiten und so beispielsweise Spracherkennung, automatisch generierte Untertitel und Sprachassistenten zu implementieren. \cite{CJLV15}

\subsection{Grobes Prinzip}\label{subsec:einleitung_nn_grobes:prinzip}
Neuronale Netze sind von der Funktionsweise menschlicher Gehirne inspiriert und versuchen dadurch eine ähnliche Lernfähigkeit zu erzielen.
Dabei werden Netze aus Neuronen simuliert, um menschliche Lernprozesse und kognitive Fähigkeiten nachzuahmen. \cite{CA18}

\bigbreak\noindent
Die Neuronen neuronaler Netze sind in Schichten organisiert, die untereinander verbunden sind, wobei die Signale von Schicht zu Schicht durch das Netz geleitet werden.
Jedes einzelne Neuron eines Netzes verarbeitet seine Eingabedaten durch die Anwendung mathematischer Funktionen und leitet das Ergebnis dann an das nächste Neuron weiter.
Die Verbindungen zwischen den Neuronen sind von variabler Stärke und sind der primäre Faktor, der beim Lernprozess verändert wird. \cite{CA18}

\bigbreak\noindent
Neuronale Netze werden trainiert, indem man sie auf einen Trainingsdatensatz anwendet und die Verknüpfungen der Neuronen basierend auf der Stärke der Abweichung vom gewünschten Ergebnis anpasst.
Der Trainingsdatensatz enthält dabei beispielhafte Eingabedaten und die dazugehörige gewünschte Ausgabe. \cite{CA18}

\bigbreak\noindent
Nach dem Trainingsvorgang kann man das Netz dann auf ihm unbekannte Daten anwenden, wobei es dann eine Ausgabe basierend auf den erlernten Mustern bildet. \cite{CA18}