\newpage
\thispagestyle{empty}
\section{Trainieren und Testen von Neuralen Netzen}\label{sec:trainierentesten}   
\begin{tcolorbox}[title={Inhalte von \textit{Trainings- und Testdaten}}]
  \begin{quotation}\noindent
    Im letzten Kapitel wird auf die Datensätze eingegangen, welche genutzt werden, um ein Modell mit Daten zu füllen und zu testen.
  \end{quotation}
  \begin{itemize}
    \item Trainings- und Testdaten
  \end{itemize}
\end{tcolorbox}

\subsection{Trainings- und Testdaten}\label{subsecsec:trainierentesten:daten}   
Zum Trainieren eines Modells wird ein Datensatz benötigt. Dieser Datensatz wird in der Regel in mindestens drei verschiedene Datensätze unterteilt: Training-, Validierung- und Testdaten.
Hier kann die Frage aufkommen, warum diese Unterteilung gemacht wird und nicht der vollständige Datensatz zum Lernen verwendet wird. 
Diese Frage sollte bei der Erklärung der Verwendungszwecke der jeweiligen Daten beantwortet werden.

\subsection{Trainingsdaten}
Ein Trainingsdatensatz ist ein Datensatz mit Beispielen. Konkret bedeutet das, dass hier eine Zuordnung zwischen Input -> Output bereits gegeben ist.
Diese Daten werden genutzt, um die Gewichte (siehe \ref*{subsec:neuronen:verknuepfung_neuronen:Weights}) des neuronalen Netzes anzupassen \cite[Seite 2ff]{CA18}.
In der Literatur wird dieser Teil oft mit 70 bis 80 Prozent von der gesamten Datenmenge bemessen.

\subsection{Validierungsdate}
Der Validierungsdatensatz dient dazu, die Leistung des Modells während des Trainings zu bewerten und die Parameter (z.B. Lernrate oder Gewichte) zu optimieren. 
Dadurch wird das Modell verbessert \cite[Seite 20]{CA18}.
In der Literatur wird dieser Teil oft mit 10 Prozent von der gesamten Datenmenge bemessen.

\subsection{Testdaten}
Mithilfe des Testdatensatzes wird die Qualität des neuronalen Netzes nach dem Training überprüft. Wichtig ist dafür, dass dieser Datensatz zuvor nicht zum Lernen verwendet wurde, 
damit diese das Ergebnis nicht verfälschen.
So kann festgestellt werden, wie das Modell auf Daten reagiert, welche nicht zuvor trainiert wurden. \cite[Seite 80f]{CA18}
In der Literatur wird dieser Teil oft mit 10 bis 15 Prozent von der gesamten Datenmenge bemessen.


Insgesamt sollte nun klar sein, warum die Datensätze aufgeteilt werden sollten. 
Wichtig anzumerken ist noch, dass die Verteilung der Datensätze variiert und so gewählt werden sollte, dass das Modell ausreichend gut trainiert und validiert wird und anschließend gründlich getestet wird.

\subsection{Problem: Overfitting}
Overfitting tritt auf, wenn das neuronale Netz zu stark an die Trainingsdaten angepasst ist und nicht gut auf neue, unbekannte Daten reagiert. 
Bei einer nur geringen Anzahl an Trainingsdaten werden nicht genug Anpassungen durchgeführt, sodass ein Lernen nach Mustern möglich wird. Die Folge davon ist, dass die Trainingsdaten auswendig gelernt werden.
Dadurch hat das neuronale Netz eine sehr hohe Genauigkeit bei den Trainingsdaten, jedoch eine sehr geringe bei dem Validierungsdatensatz \cite[Seite 25]{CA18}. 